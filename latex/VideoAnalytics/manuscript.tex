\documentclass[conference]{IEEEtran}
\IEEEoverridecommandlockouts
% The preceding line is only needed to identify funding in the first footnote. If that is unneeded, please comment it out.
\usepackage{amsthm}
\usepackage{caption}
\usepackage{url}
\usepackage{amssymb}
\usepackage{graphicx}
%\usepackage{floatrow}
\usepackage{tabularx}
\usepackage{float}
\usepackage{amsmath}
\usepackage{mathtools}
\usepackage{latexsym}
\usepackage{amsfonts}
\usepackage{amssymb}
\usepackage[ruled]{algorithm}
\usepackage{cite}
\usepackage{algorithmic}
\usepackage{xspace}
\usepackage{epstopdf}
\usepackage[caption=false]{subfig}
\usepackage[T1]{fontenc}
%\usepackage{authblk}
\usepackage{url}
\usepackage{rotating}
\usepackage[usenames,dvipsnames]{color}
\usepackage{color}
\usepackage{colortbl}
\usepackage{comment}
\usepackage{mathrsfs}
\usepackage{makecell}

\usepackage{multicol}
\usepackage{multirow}
\usepackage{makecell}
\usepackage{bm}
\usepackage{paralist}
\usepackage{enumitem}
\usepackage{eqparbox} 
\usepackage{booktabs}

% \usepackage{hyperref}


\usepackage{fancyhdr} % 添加页眉页脚

\usepackage{subfigure}
% \usepackage{ulem}

% \usepackage{accents}
% \usepackage{statex}
% \makeatletter
% \def\wideubar{\underaccent{{\cc@style\underline{\mskip10mu}}}}
% \def\Wideubar{\underaccent{{\cc@style\underline{\mskip8mu}}}}
% \makeatother

% \makeatletter
% \newcommand\dlmu[2][0.5cm]{\hskip1pt\underline{\hb@xt@ #1{\hss#2\hss}}\hskip3pt}
% \makeatother
\usepackage{accents}
\newcommand{\ubar}[1]{\underaccent{\bar}{#1}}

% \renewcommand{\algorithmiccomment}[1]{\hfill\eqparbox{COMMENT}{\# #1}} 
\renewcommand{\algorithmiccomment}[1]{\eqparbox{COMMENT}{\rhd #1}} 

\newcommand{\ie}{\textit{i.e.}\xspace}
\newcommand{\etal}{\textit{et al.}\xspace}
\newcommand{\eg}{\textit{e.g.}\xspace}

\newtheorem{theorem}{Theorem}
\newtheorem{lemma}{Lemma}
\newtheorem{proposition}{Proposition}
\newtheorem{definition}{Definition}
\newtheorem{assump}{Assumption}
\newtheorem{observe}{Observation}
\DeclareMathOperator*{\argmax}{arg\,max}
% \def\BibTeX{{\rm B\kern-.05em{\sc i\kern-.025em b}\kern-.08em
%     T\kern-.1667em\lower.7ex\hbox{E}\kern-.125emX}}

\begin{document}

% \title{Multi-objective-aware Data Collection in UAV Crowdsensing: A Multi-objective Bandit Approach}
\title{Video Analytics + crowdsourcing + vehicular networks}


\maketitle

\thispagestyle{fancy} % IEEE模板在\maketitle后会自动声明\thispagestyle{plain},
                            % 导致第一页什么都没有。所以得把plain更改为fancy
\lhead{} % 页眉左,需要东西的话就在{}内添加
\chead{} % 页眉中
\rhead{} % 页眉右
\lfoot{} % 页眉左
\cfoot{} % 页眉中
\rfoot{\thepage} %页眉右,\thepage 表示当前页码
\renewcommand{\headrulewidth}{0pt} %改为0pt即可去掉页眉下面的横线
\renewcommand{\footrulewidth}{0pt} %改为0pt即可去掉页脚上面的横线
\pagestyle{fancy}
\rfoot{\thepage}


\begin{abstract}
Abstract goes here.
\end{abstract}

\begin{IEEEkeywords}
Key word1, Key word2, Key word3, Key word4
\end{IEEEkeywords}

\section{Introduction}
With the rapid development of smart cities and intelligent transportation systems, vehicular networks have emerged as a crucial component to facilitate the acquisition and analysis of real-time data. Among various applications, video analytics has gained significant attention due to its potential to improve road safety, traffic management, and urban surveillance. However, video analytics tasks are often resource-intensive and require substantial computational power and network bandwidth. As a result, efficiently processing these tasks in vehicular networks presents significant challenges.

Edge computing has become a promising solution to address these challenges by offloading computation-intensive tasks from resource-constrained vehicles to edge servers with stronger processing capabilities. Using edge computing, delay and bandwidth consumption in video analytics tasks can be significantly reduced, improving system performance and task completion rates. In addition, crowdsourcing systems have been widely adopted to utilize diverse vehicular resources by assigning tasks to available vehicles or edge devices.

In this paper, we propose an edge-assisted crowdsourcing system for video analytics tasks in vehicular networks. Our system aims to allocate tasks efficiently to minimize energy consumption while ensuring task completion. The proposed system leverages the computational capabilities of nearby edge servers and vehicle-mounted devices to process video data in a distributed manner.

To achieve efficient task allocation, we model the system by incorporating the characteristics of tasks, workers, and the platform. Each task arriving in the system has specific requirements, such as data size, resolution, and frame rate, which directly impact energy consumption. Workers are modeled as devices with distinct transmission and processing energy profiles, ensuring a detailed representation of real-world energy constraints. The crowdsourcing platform acts as a central controller that assigns tasks to workers while maximizing its own profit by balancing revenue from completed tasks and payment to workers.

The core contribution of this paper lies in the comprehensive modeling framework that integrates multiple factors, including energy consumption, task requirements, and platform revenue. We develop a flexible task allocation strategy that dynamically assigns tasks based on system conditions, ensuring optimal performance. Through extensive simulations, we demonstrate the effectiveness of our proposed system in achieving energy-efficient task allocation while maintaining high task completion rates and maximizing platform profit.

The remainder of this paper is organized as follows: Section II presents the related work, Section III presents the system model, including task modeling, worker modeling, and platform modeling. Section IV introduces the proposed task allocation strategy. Section V evaluates the system's performance through extensive experiments, and Section VI concludes the paper with future research directions.



\section{Related Work}
xxx

\section{System Model}
In this paper, we consider an edge-assisted crowdsourcing system for video analytics tasks in vehicular networks. {\color{blue}We present modeling of tasks, workers and the platform in the system as follows. Then, we formulate the problem studied in this paper.}


\subsection{Task modeling}
% In the system, tasks are divided into multiple time slots. Each time slot \( t \) is assigned to a specific worker or a set of workers, denoted as \( \mathcal{W}^t = \{ w_1^t, w_2^t, \dots, w_n^t \} \), where each \( w_i^t \) represents a worker assigned to the task in time slot \( t \). Each worker is responsible for completing the assigned task, such as video analysis or data collection, within the corresponding time slot. 
{\color{blue}We consider video analytics tasks that arrive in our system over time. For convenience, we assume the system is time-slotted, which means the system operates slot-by-slot.} 



The concept of a slot helps to structure the time-based allocation of tasks and ensures that the workload is distributed efficiently among workers. A slot represents a discrete unit of time during which the workers in \( \mathcal{W}^t \) must perform the assigned tasks and upload their results. However, not all workers in \( \mathcal{W}^t \) are necessarily assigned a task. Each time slot \( t \) may only have a subset of workers assigned tasks, depending on the system's task allocation strategy.

Furthermore, each time slot \( t \) contains \( M \) incoming tasks, denoted as \( \mathcal{T}^t = \{ T_1^t, T_2^t, \dots, T_M^t \} \), where each \( T_i^t \) represents a task arriving at time slot \( t \). These tasks could be video analysis tasks, data collection requests, or other computational tasks. The tasks in \( \mathcal{T}^t \) are to be allocated to the workers in \( \mathcal{W}^t \), where some tasks may not require a worker to be assigned if no suitable worker is available.

To model the assignment of tasks to workers, we introduce a binary variable \( x_{i,j}^t \), which is defined as follows:
\[
x_{i,j}^t = 
\begin{cases} 
1, & \text{if worker } w_i^t \text{ is assigned to task } T_j^t \text{ at time slot } t, \\
0, & \text{otherwise}.
\end{cases}
\]
The binary variable \( x_{i,j}^t \) indicates whether worker \( w_i^t \) is assigned to complete task \( T_j^t \) in time slot \( t \). If \( x_{i,j}^t = 1 \), worker \( w_i^t \) will perform task \( T_j^t \); otherwise, the worker will not be assigned that task during that time slot. This modeling allows for flexible allocation of tasks to workers, where some tasks may remain unassigned if no suitable worker is available, or if the system decides not to allocate them during that specific time slot.

\subsection{Worker modeling}
In the edge-assisted crowdsourcing system for video analysis tasks, we define a worker as a device, a vehicle, or any node with processing power. Their task is to receive tasks assigned by the system, collect video data, analyze video content, and upload results. These workers are equipped with various energy-consuming components, which include transmission and processing units. 

\subsubsection{Context Vector Definition and Normalization}
In our system, task and worker characteristics are modeled as a context vector $\mathbf{z}_i^t \in \mathbb{R}^d$, where each dimension represents a relevant factor influencing task execution. These factors include:
\begin{itemize}
    \item \textbf{Driving speed}: The speed at which the worker (e.g., vehicle) is traveling, which can affect the task execution time and data transmission. The unit is meters per second (m/s).
    \item \textbf{Bandwidth}: The communication bandwidth available to the worker, impacting the data transfer rate for task execution. The unit is megabits per second (Mbps).
    \item \textbf{Processor performance}: The computational power available to the worker, influencing the speed and efficiency of task processing. This is often measured in gigahertz (GHz) or FLOPS (floating-point operations per second).
    \item \textbf{Physical location (distance)}: The distance from the worker to the nearest task upload location, affecting latency and energy consumption. The unit is meters (m).
    \item \textbf{Task type}: The type of task being assigned, which determines its computational complexity and resource requirements. It is represented as a categorical variable or indexed accordingly.
    \item \textbf{Data size}: The size of the data that needs to be processed, which affects both processing time and energy consumption. The unit is megabytes (MB) or gigabytes (GB).
    \item \textbf{Weather}: Environmental conditions such as weather, which can be directly observed by the platform and may affect task performance, such as data transmission reliability or worker mobility. This is represented as a categorical or numerical value.
\end{itemize}

All variables in the context vector $\mathbf{z}_i^t$ are normalized to a predefined range to fit the context space for partitioning. This ensures consistency in the dimensions and allows the context space to be adaptively partitioned into cells. Specifically, the variables are normalized as follows:
\begin{itemize}
    \item \textbf{Driving speed, bandwidth, and processor performance} are normalized to fit within a fixed range based on typical system specifications.
    \item \textbf{Physical location (distance)} is normalized based on the maximum distance observed in the system.
    \item \textbf{Task type} is normalized into a categorical index.
    \item \textbf{Data size} is normalized based on the maximum size encountered in the dataset.
    \item \textbf{Weather} is normalized based on predefined weather categories or numeric scales.
\end{itemize}

The context space $\mathcal{Z}$ is adaptively partitioned into cells, each representing a subset of arms, where each arm corresponds to a worker-task combination with similar performance characteristics. Initially, the entire context space $\mathcal{Z}$ is treated as a single coarse region, providing a general approximation of all arms. As the system progresses and receives feedback from task outcomes, these regions are incrementally refined to capture areas with higher performance accuracy.

The adaptive refinement of the context space allows the system to focus on promising areas, accelerating the identification of high-reward arms and ensuring faster convergence to optimal or near-optimal solutions.

To capture the energy consumption of a worker, we divide it into two primary components: transmission energy consumption and video analysis energy consumption.

\subsubsection{Transmission Energy Consumption}
The transmission energy consumption refers to the energy spent by the worker on downloading video data from the edge server or transmitting video data to the edge server or other network nodes. This energy is primarily determined by the data size, transmission distance, and communication protocol used. The transmission energy for a worker \(w_i^t\) on task \(T_j^t\), denoted as \( E_{\text{trans}, i,j}^t \), can be expressed as:
\[
E_{\text{trans}, i,j}^t =  \alpha_{\text{trans}} \cdot D_j^t \cdot \left( \frac{d_i}{d_0} \right)^\beta \cdot x_{i,j}^t
\]
where:
\begin{itemize}
    \item \( D_j^t \) is the amount of data transmitted for task \( j \) during time slot \( t \), which can be computed as:
    \item \( d_i \) is the transmission distance from worker \( w_i^t \) to the receiving node,
    \item \( d_0 \) is a reference distance (1 meter in our experiments),
    \item \( \alpha_{\text{trans}} \) is a constant that represents the energy per unit of data transmitted at the reference distance,
    \item \( \beta \) is the path loss exponent, which varies depending on the communication environment (e.g., urban, rural).
    \item \(x_{i,j}^t\) is whether task \(j\) is allocated to worker \(i\) in time slot t
\end{itemize}
To model \( D_j^t \), we consider two key factors: frame rate and resolution. Specifically, the amount of data transmitted for task \( j \) during time slot \(t\) can be modeled as:
\[
D_j^t = f_j \cdot R_j 
\]
where:
\begin{itemize}
    \item \( f_j \) is the frame rate of video for task \( j \) (frames per second),
    \item \( R_j \) is the resolution of each frame for task \( j \) (in bits per pixel),
\end{itemize}

The energy consumption increases with the amount of data to be transmitted and the distance to the receiving node, which is crucial for efficient task allocation in terms of energy.

\subsubsection{Video Analysis Energy Consumption}
The energy consumption for video analysis refers to the energy expended by the worker to process video data, including tasks such as video decoding, feature extraction, object detection, and other computational operations. The energy consumed by worker \( w_i^t \) for analyzing the video data during time slot \( t \), denoted as \( E_{\text{analyze}, i}^t \), can be expressed as:

\[
E_{\text{analyze}, i, j}^t = \alpha_{\text{analyze}} \cdot P_{\text{total}, i}^t \cdot x_{i,j}^t
\]

where:
\begin{itemize}
    \item \( P_{\text{total}, i}^t \) is the total power consumed by worker \( w_i^t \) while analyzing video data in time slot \( t \),
    \item \( \alpha_{\text{analyze}} \) is a constant representing the energy consumption per unit of processing power per unit of time,
    \item \(x_{i,j}^t\) is whether task \(j\) is allocated to worker \(i\) in time slot \(t\).
\end{itemize}

To model \( P_{\text{total}, i}^t \), we consider two key factors: task type and frame rate. Specifically, the total power consumed by worker \( w_i^t \) can be modeled as:

\[
P_{\text{total}, i}^t = \alpha_{\text{task}, i} \cdot P_{\text{base}, i}^t \cdot \frac{f_i^t}{f_0}
\]

where:
\begin{itemize}
    \item \( P_{\text{base}, i}^t \) is the baseline power consumed by worker \( w_i^t \) for performing basic video analysis tasks (e.g., decoding), 
    \item \( f_i^t \) is the frame rate of the video being analyzed during time slot \( t \),
    \item \( f_0 \) is a reference frame rate (30 fps in our experiments), 
    \item \( \alpha_{\text{task}, i} \) is a constant specific to the task type, representing the scaling factor based on the complexity of the task (e.g., object detection or action recognition).
\end{itemize}

This energy consumption model can be used to estimate the expected power usage based on worker's context before the task execution.


\subsubsection{Total Energy Consumption}
The total energy consumed by worker \( w_i^t \) during time slot \( t \) can be computed by summing the transmission and analysis energy consumption:
\[
E_{i}^t = \sum _{j=1} ^M (E_{\text{trans}, i,j}^t + E_{\text{analyze}, i,j}^t)
\]
where:
\begin{itemize}
    \item \( E_{\text{analyze}, i, j}^t \) denotes the estimated computational energy required by worker \( w_i^t \) to execute the video analysis operations associated with task \( T_j^t \),
    \item \( E_{\text{trans}, i, j}^t \) denotes the estimated communication energy required by worker \( w_i^t \) to upload or download video data associated with task \( T_j^t \),
\end{itemize}


\subsection{Platform modeling}
The crowdsourcing platform acts as an intermediary that assigns video analysis tasks to workers and receives rewards from task requesters. The platform's profit during time slot \( t \), denoted as \( \Pi^t \), is determined by the revenue obtained from completed tasks and the costs paid to workers.

\subsubsection{Platform Revenue}
The platform earns revenue for each successfully completed task. The revenue obtained from task \( j \) during time slot \( t \), denoted as \( R_j^t \), can be formulated as:
\[
R_j^t = p_j \cdot Q_j^t
\]
where:
\begin{itemize}
    \item \( p_j \) is the payment offered by the requester for unit quality of task \( j \),
    \item \( Q_j^t \) is the quality of the completed task \( j \) during time slot \( t \), which depends on the accuracy, latency, and completeness of the video analysis task.
\end{itemize}

%这一段是否需要在cost中加入对质量的权衡(已删除)
\subsubsection{Worker Reward}
The platform needs to compensate workers for their computational resources and energy consumption. The reward paid to worker \( w_i^t \) for completing tasks during time slot \( t \), denoted as \( C_i^t \), can be modeled as:
\[
C_i^t =   \lambda E_i^t
\]
where:
\begin{itemize}
    \item \( \lambda \) is the compensation rate per unit energy consumed.
\end{itemize}

\subsubsection{Platform Profit}
The platform's total profit during time slot \( t \), denoted as \( \Pi^t \), can be expressed as:
\[
\Pi^t = \sum_{j=1}^{M} R_j^t - \sum_{i=1}^{N} C_i^t
\]
where:
\begin{itemize}
    \item \( M \) is the total number of tasks,
    \item \( N \) is the total number of workers.
\end{itemize}

% 连续性假设
% semi-bandit 假设
% arm独立性假设
% 每个时刻都会到来新的arm,volatile假设
% bounded 奖励

\subsection{Assumptions}

To support the design and theoretical analysis of the proposed learning-based task assignment algorithm, we make the following assumptions:

\begin{itemize}
    \item \textbf{(A1) Lipschitz continuity of the expected reward:}  
    The expected platform reward function \( Q(\mathbf{z}_i^t) \) is assumed to be Lipschitz continuous over the context space \( \mathcal{Z} \). That is, there exists a constant \( L > 0 \) such that for any two context vectors \( \mathbf{z}_i^t, \mathbf{z}_j^t \in \mathcal{Z} \), the following holds:
    \[
    \left| Q(\mathbf{z}_i^t) - Q(\mathbf{z}_j^t) \right| \leq L \| \mathbf{z}_i^t - \mathbf{z}_j^t \|_2.
    \]
    This assumption ensures that similar arms yield similar expected rewards, enabling effective generalization and partitioning of the context space.


    \item \textbf{(A2) Conditional independence of task outcomes:}  
    Given the context vectors \( \mathbf{z}_i^t \), the outcomes (i.e., task quality or reward) of different arms are conditionally independent. 
    Formally,
    
    \[
    \mathbb{P}(Q(\mathbf{z}_1^t), Q(\mathbf{z}_2^t), \dots, Q(\mathbf{z}_N^t) \mid \mathbf{z}_1^t, \dots, \mathbf{z}_N^t) = \prod_{i=1}^{N} \mathbb{P}(Q(\mathbf{z}_i^t) \mid \mathbf{z}_i^t).
    \]

    \item \textbf{(A3) Semi-bandit feedback:}  
    After assigning tasks, the platform receives individual feedback on the task quality \( Q(\mathbf{z}_i^t) \) for each selected arm. The observed reward is assumed to depend solely on the corresponding context vector \( \mathbf{z}_i^t \).


    \item \textbf{(A4) Bounded reward and cost:}  
    The expected platform reward \( Q(\mathbf{z}_i^t) \) and the corresponding worker cost \( C_i^t \) are both assumed to be bounded. Specifically,
    \[
    Q(\mathbf{z}_i^t) \in [0, Q_{\max}], \quad C_i^t \in [0, C_{\max}],
    \]
    where \( Q_{\max} \) and \( C_{\max} \) are known positive constants. This assumption guarantees that the resulting profit is well-defined and avoids unbounded variance in the learning process.

    \item \textbf{(A5) Dynamic worker availability and feasibility:}  
The set of available workers varies across time slots in a dynamic and uncertain manner. At each time slot \( t \), we assume that the number of available workers \( N \) exceeds the number of tasks \( M \), i.e., \( N > M \). This ensures the feasibility of task assignment under system constraints and allows one-to-one matching between tasks and workers.
\end{itemize}


\subsection{Problem formulation}
The goal of the crowdsourcing platform is to maximize its profit by optimally assigning video analysis tasks to workers while ensuring task quality, managing energy consumption, and satisfying system constraints.

The optimization problem is formulated as follows:

\begin{align}
    \max_{\mathbf{x}} \quad & \Pi^t = \sum_{j=1}^{M} R_j^t - \sum_{i=1}^{N} C_i^t  \tag{1} \\
    \text{s.t.} \quad 
    & \sum_{j=1}^{M} x_{i,j}^t \leq 1, \quad \forall i, t  \tag{2} \\
    & \sum_{i=1}^{N} x_{i,j}^t \geq 1, \quad \forall j, t  \tag{3} \\
    & x_{i,j}^t \in \{0,1\}, \quad \forall i,j,t  \tag{4} \\
    & E_{\text{analyze}, i}^t + E_{\text{trans}, i}^t \leq E_i^{\max}, \quad \forall i, t  \tag{5} \\
    & \sum_{t=1}^{T_j^{\max}} \sum_{i=1}^{N} x_{i,j}^t \cdot Q_{i,j}^t \geq Q_j^{\text{req}}, \quad \forall j  \tag{6} \\
    & \sum_{i=1}^{N} \sum_{j=1}^{M} x_{i,j}^t \cdot D_j^t \leq B^{\max}, \quad \forall t  \tag{7}
\end{align}

\subsubsection{Explanation of Constraints}
\begin{itemize}
    \item \textbf{Objective function (1)}: Maximizes the platform’s profit \( \Pi^t \), defined as the total revenue \( R_j^t \) from assigned tasks minus the total cost \( C_i^t \) of worker payments.
    \item \textbf{Constraint (2)}: Ensures that each worker \( w_i \) can handle at most one task at any given time \( t \).
    \item \textbf{Constraint (3)}: Ensures that each task \( j \) is assigned to at least one worker.
    \item \textbf{Constraint (4)}: Defines \( x_{i,j}^t \) as a binary decision variable, indicating whether worker \( i \) is assigned to task \( j \) at time \( t \).
    \item \textbf{Constraint (5)}: Ensures that the total energy consumed by a worker \( i \) (including computing and transmission energy) does not exceed its available energy \( E_i^{\max} \).
    \item \textbf{Constraint (6)}: Ensures that the total accumulated task quality \( Q_{i,j}^t \) over the allowed time slots meets the minimum required quality \( Q_j^{\text{req}} \).
    \item \textbf{Constraint (7)}: Ensures that the total data transmission does not exceed the maximum available bandwidth \( B^{\max} \).
\end{itemize}

The problem formulation ensures that the system efficiently allocates tasks while considering energy consumption, bandwidth limitations.

% reward 和 cost在上下文空间的方法中并没有详细地进行描述,可能是因为当时地状态被直接映射到上下文空间中,方法根据context space对应的 profit 函数 (进而可以直接得到最终选取的结果),所以reward和cost对应的计算被囊括到了context space的细分过程中。【那reward和cost的具体公式还有必要在models中去具体说明吗】

\noindent
\textbf{Notation.} To facilitate understanding of the proposed system and algorithmic framework, we summarize the key mathematical symbols and their meanings in Table~\ref{tab:notations}. These notations are used throughout the paper in both the system modeling and the method design sections.

% 表格现在太宽了放不下
\begin{table*}[t]
\centering
\scriptsize
\caption{Notations Table}
\label{tab:notations}
\begin{tabular}{llll}
\toprule
\textbf{Symbol} & \textbf{Description} & \textbf{Unit} & \textbf{Category} \\
\midrule
$t$ & Time slot index & - & System Model \\
$T$ & Total number of time slots & - & System Model \\
$N$ & Number of available workers at time $t$ & - & Task/Worker Model \\
$M$ & Number of incoming tasks at time $t$ & - & Task Model \\
$w_i^t$ & Worker $i$ at time $t$ & - & Worker Model \\
$T_j^t$ & Task $j$ at time $t$ & - & Task Model \\
$x_{i,j}^t$ & Assignment variable (1 if $w_i^t$ is assigned to $T_j^t$) & Binary & Task Assignment \\
$\mathbf{z}_{i,j}^t$ & Context vector for arm $(i,j)$ & $\mathbb{R}^d$ & Context Model \\
$\mathcal{Z}$ & Context space & - & Context Model \\
$\mathcal{C}_i^t$ & Context cell containing $\mathbf{z}_{i,j}^t$ & - & Context Model \\
$\hat{Q}(\mathbf{z}_{i,j}^t)$ & Estimated task quality of $\mathbf{z}_{i,j}^t$ & - & Context Model \\
$Q(\mathbf{z}_{i,j}^t)$ & True reward of $\mathbf{z}_{i,j}^t$ & - & Reward Model \\
$c^{t}(\mathcal{C}_i^t)$ & Confidence term for cell $\mathcal{C}_i^t$ & - & Bandit Model \\
$\Delta(\mathcal{C}_i^t)$ & Hierarchical correction term for cell $\mathcal{C}_i^t$ & - & Bandit Model \\
$S(\mathcal{C}_i^t)$ & Number of times cell $\mathcal{C}_i^t$ has been split & - & Bandit Model \\
$p_j$ & Payment per unit quality for task $j$ & - & Platform Model \\
$C_{i,j}^t$ & Estimated energy cost for $w_i^t$ to perform $T_j^t$ & - & Platform Model \\
$\hat{\Pi}_{i,j}^t$ & Estimated profit of assigning $w_i^t$ to $T_j^t$ & - & Profit Model \\
$E_{\text{trans},i,j}^t$ & Transmission energy of worker $i$ for task $j$ at time $t$ & Joules & Energy Model \\
$E_{\text{analyze},i,j}^t$ & Analysis energy of worker $i$ for task $j$ at time $t$ & Joules & Energy Model \\
$E_i^t$ & Total energy consumption of worker $i$ at time $t$ & Joules & Energy Model \\
$C_i^t$ & Reward (cost) paid to worker $i$ for energy consumption & - & Platform Cost \\
$R_j^t$ & Platform revenue from task $j$ at time $t$ & - & Platform Revenue \\
$\Pi^t$ & Platform profit at time $t$ & - & Platform Profit \\
$D_j^t$ & Amount of video data for task $j$ & MB or GB & Task Model \\
$f_j$ & Frame rate for task $j$ & fps & Task Model \\
$R_j$ & Resolution per frame for task $j$ & bits/pixel & Task Model \\
$d_i$ & Distance from worker $i$ to data upload location & m & Worker Model \\
$\alpha_{\text{trans}}$ & Transmission energy coefficient & - & Energy Model \\
$\alpha_{\text{analyze}}$ & Analysis energy coefficient & - & Energy Model \\
$P_{\text{base}, i}^t$ & Baseline power of worker $i$ during analysis & W & Energy Model \\
$P_{\text{total}, i}^t$ & Total power of worker $i$ for analysis & W & Energy Model \\
$N_{cell}^t$& Number of partitions (cells) in context space at time $t$ & - & Complexity Analysis \\
\bottomrule
\end{tabular}
\end{table*}

\input{}





\section{Our Method}

To address the challenges of dynamic worker availability and energy-aware task assignment in vehicular edge-assisted crowdsourcing, we design an adaptive learning-based task allocation strategy. This method dynamically estimates the performance of workers based on their contextual attributes and historical outcomes, while optimizing the assignment of tasks to minimize system-wide energy consumption and ensure successful task execution.

\subsection{Adaptive Contextual Space Modeling}

\subsubsection{Contextual Bandit Base Settings}
In our system, each available worker at time slot $t$ is viewed as an \textit{arm candidate}, represented by a context vector $\mathbf{z}_i^t \in \mathbb{R}^d$. This context captures the worker’s key attributes.

We define an \textit{arm} as the abstract representation of a worker-task pair in a specific system state, where each arm is associated with a context vector \( \mathbf{z}_{i,j}^t \in \mathcal{Z} \), encoding the relevant attributes of the pair. At each time slot, the set of all such arm candidates forms a dynamic set, determined by the currently available workers and incoming tasks. These arms are embedded in a continuous \textit{context space} \( \mathcal{Z} \), which serves as the domain for modeling their expected performance.

To facilitate efficient estimation and learning, we partition the context space \(\mathcal{Z}\) into hierarchical regions that are adaptively refined over time. Each cell is associated with a local empirical model of expected task quality, which is updated as new outcomes are observed. This structure enables the system to distinguish between high-performing and low-performing regions, allowing more accurate selection of arms for execution. Moreover, by concentrating exploration on promising areas of the context space, the system can more rapidly converge to high-reward arms, significantly accelerating the identification of optimal or near-optimal assignments.

Given the context vectors \( \mathbf{z}_i^t \), the system computes a value function \( \hat{Q}(\mathbf{z}_i^t) \) for each arm, which reflects the expected reward associated with that arm. This value function captures the overall performance of the arm, implicitly considering factors such as energy consumption and task execution quality. The value function \( \hat{Q}(\mathbf{z}_i^t) \) is learned using a contextual bandit algorithm and gradually converges to the true task quality estimate \( E(Q(\mathbf{z}_i^t)) \) over time.


\subsubsection{Value Function}
To compute the value function, we define a function \( \hat{Q}(\mathbf{z}_i^t) \) that incorporates the confidence interval and hierarchical correction term. The confidence term ensures that the estimate improves with more feedback, and the hierarchical correction term adjusts the estimate based on the granularity of the context space partition. Specifically, the context vector \( \mathbf{z}_i^t \) is mapped to a specific cell \( \mathcal{C}_i^t \) in the context space \( \mathcal{Z} \), and the corresponding reward estimate is computed based on the performance of the arms within that cell. The function \( \hat{Q}(\mathbf{z}_i^t) \) can be expressed as:
\[
\hat{Q}(\mathbf{z}_i^t) = \hat{\mu}^{t-1}(\mathcal{C}_i^t) + c^{t-1}(\mathcal{C}_i^t) + \Delta(\mathcal{C}_i^t)
\]
where:
\begin{itemize}
    \item \( \hat{\mu}^{t-1}(\mathcal{C}_i^t) \) is the empirical estimate of the task quality for the cell \( \mathcal{C}_i^t \) containing vector \( \mathbf{z}_i^t \) at the previous time slot \( t-1 \), based on the historical observations of task completions within that cell;
    \item \( c^{t-1}(\mathcal{C}_i^t) \) is the confidence term based on previous observations at \( t-1 \), reflecting the number of times arms have been selected within the cell \( \mathcal{C}_i^t \);
    \item \( \Delta(\mathcal{C}_i^t) \) represents the hierarchical error term, which accounts for the granularity of the context space partition and is adjusted based on how many times the cell \( \mathcal{C}_i^t \) has been split.
\end{itemize}

In the context of the system, each arm is represented by a context vector \( \mathbf{z}_i^t \), which falls within a specific cell \( \mathcal{C}_i^t \) of the context space \( \mathcal{Z} \). The performance of arms within each cell is observed over time, and the reward \( \hat{Q}(\mathbf{z}_i^t) \) is updated by aggregating the task quality results reported by the workers. Once a cell is selected and the corresponding tasks are executed, feedback is collected and used to refine the estimates for all observed arms within that cell.

The average task quality for cell \( \mathcal{C}_i^t \) is computed as:

\[
\hat{\mu}^{t-1}(\mathcal{C}_i^t) = 
\begin{cases} 
0 & \text{if } t = 0 \\
\frac{\sum_{\text{selected arms in } \mathcal{C}_i^{t-1}} Q(\mathbf z_i^{t-1})}{N(\mathcal{C}_i^{t-1})} & \text{if } t > 0
\end{cases}
\]

where:
\begin{itemize}
    \item \( N(\mathcal{C}_i^{t-1}) \) is the total number of selections of arms within cell \( \mathcal{C}_i^t \) until the \( t-1 \) time slot. 
    \item \( Q(\mathbf z_i^{t-1})\) represents the task quality reported by the worker for the selected arms within that cell.
\end{itemize}

The confidence term \( c^{t-1}(\mathcal{C}_i^t) \) is computed as:
\[
c^{t-1}(\mathcal{C}_i^t) := \sqrt{\frac{\ln T}{N(\mathcal{C}_i^{t-1})}}
\]
where:
\begin{itemize}
    \item \( T \) is the total number of time slots in the system up to the present.
\end{itemize}

The hierarchical error term \( \Delta(\mathcal{C}_i^t) \) accounts for the granularity of the context space partition. When a cell is selected too frequently and its performance becomes highly reliable, the cell is split to allow finer granularity in the context space, thus providing more accurate task performance estimates. 
\[
\Delta(\mathcal{C}_i^t) = \rho ^h}
\]
At refinement level $h$, each cell is recursively divided into two subcells along every dimension, resulting in subcells with diameter exactly $\rho^h$ for some fixed constant $0 < \rho < 1$. Specifically, we use a binary splitting approach where \(\rho = 0.5\), meaning that at each refinement step, each cell is divided into two equal subcells, and the diameter of each subcell decreases by a factor of \(0.5\) at each level.

\subsubsection{Mechanism of Division and Refine}
In our system, the context space is initially partitioned into multiple broad cells, each representing a region of arms. Each cell is associated with an empirical estimate of the task quality. Over time, as more feedback is gathered, the system adapts the context space by refining the cells. Specifically, when a cell is selected too frequently and its performance becomes sufficiently stable, the cell will undergo a split to create smaller sub-regions for more granular task quality estimation.

The splitting mechanism is triggered when the confidence term \( c^t(\mathcal{C}_i^t) \) satisfies the condition:
\[
c^t(\mathcal{C}_i^t) \leq \rho^h,
\]
Once the confidence term falls below this threshold, the cell is considered sufficiently refined, and the splitting process is triggered to divide the cell into smaller subcells for further exploration.

Upon satisfying the splitting condition, the cell \( \mathcal{C}_i^t \) is divided into \( N \) smaller sub-cells, which are then added to the set of active regions for future exploration. This process helps the system to focus on promising areas of the context space that show higher expected rewards, while avoiding over-exploration in areas where sufficient performance information has already been gathered. 

This adaptive refinement mechanism ensures that the system progressively improves the precision of task quality estimates by focusing computational resources on more promising regions of the context space.

% 这里已经知道如何估计并更新赌博机,现在我们得到了每个worker对于每个task的完成质量估计,问题转化为worker的分配问题

\subsection{Profit-Optimal Assignment Modeling}

After estimating the task quality \( \hat{Q}(\mathbf{z}_{i,j}^t) \) and energy cost \( C_i^t \) for each arm at time slot \( t \), we define the estimated platform profit as:

\[
\hat{\Pi}_{i,j}^t = p_j \cdot \hat{Q}(\mathbf{z}_{i,j}^t) - C_i^t
\]
This results in a profit matrix $\hat{\Pi}^t \in \mathbb{R}^{N \times M}$.

The task assignment problem is formulated as a maximum-weight bipartite matching between the set of workers and the set of tasks. Each worker $w_i$ can be matched to at most one task $T_j$, and vice versa. The goal is to find a binary assignment matrix $x_{i,j}^t \in \{0, 1\}$ that maximizes the total profit:
\[
\max_{x_{i,j}^t} \sum_{i=1}^N \sum_{j=1}^M x_{i,j}^t \cdot \hat{\Pi}_{i,j}^t
\]
subject to:
\begin{align*}
& \sum_{j=1}^M x_{i,j}^t \leq 1, \quad \forall i \quad \text{(one task per worker)} \\
& \sum_{i=1}^N x_{i,j}^t = 1, \quad \forall j \quad \text{(each task must be assigned)} \\
& x_{i,j}^t \in \{0,1\}
\end{align*}

We solve this assignment problem using the Hungarian algorithm, which computes the optimal one-to-one matching in polynomial time $\mathcal{O}(N^3)$. Since we assume that the number of workers $N$ exceeds the number of tasks $M$, we pad the profit matrix with $N - M$ virtual tasks (columns with zero profit) to obtain a square matrix. This ensures feasibility for the algorithm and enables optimal task allocation based on the estimated reward and cost.





\subsection{Algorithm Overview}

Based on the modeling described in the previous section, we now present the full procedure of our adaptive task allocation strategy. At each time slot, the algorithm estimates the contextual quality and profit, solves the profit-maximizing assignment using the Hungarian algorithm, and updates the contextual models based on observed feedback.

To implement our adaptive task assignment framework, we develop an online learning algorithm that iteratively observes the system state and selects an arm allocation strategy that maximizes the platform’s profit. At each time slot, the platform collects the context vectors of available arms and estimates the expected reward (i.e., task quality) and energy consumption for each arm using contextual bandit-based modeling.

Rather than selecting arms solely based on estimated quality, our method jointly considers the estimated reward and cost to compute the expected profit of each arm. We formulate a constrained assignment problem to find the optimal mapping over arms that maximizes the total profit while satisfying system constraints. The full procedure is summarized in Algorithm~\ref{alg:adaptive}.


% 这里插入算法整体流程图

\begin{algorithm}[H]
\caption{Adaptive Contextual Task Allocation Strategy}
\label{alg:adaptive}
\begin{algorithmic}[1]
\FOR{each time slot $t = 1, 2, \dots, T$}
    \STATE Observe available workers $\mathcal{W}^t$ and incoming tasks $\mathcal{T}^t$
    \FOR{each worker $w_i^t \in \mathcal{W}^t$}
        \STATE Construct context vector $\mathbf{z}_i^t$
        \STATE Identify cell $\mathcal{C}_i^t$ containing $\mathbf{z}_i^t$
        \STATE Compute estimated task quality $\hat{Q}(\mathbf{z}_i^t)$
        \STATE Compute estimated cost $C_{i,j}^t$ for each task $T_j^t$
        \STATE Compute estimated profit $\hat{\Pi}_{i,j}^t = p_j \cdot \hat{Q}(\mathbf{z}_i^t) - C_{i,j}^t$
    \ENDFOR
    \STATE Solve the assignment problem:
        \begin{itemize}
            \item Decision variable: $x_{i,j}^t \in \{0,1\}$
            \item Objective: $\max \sum_{i,j} x_{i,j}^t \cdot \hat{\Pi}_{i,j}^t$
            \item Subject to: constraints on energy, bandwidth, one-task-per-worker, etc.
        \end{itemize}
    \STATE Assign selected arms according to optimal $x_{i,j}^t$
    \STATE Receive feedback $Q(\mathbf{z}_{i,j}^t)$ for each selected arm
    \FOR{each selected cell $\mathcal{C}_i^t$}
        \STATE Update $\hat{\mu}(\mathcal{C}_i^t)$, $c(\mathcal{C}_i^t)$
        \IF{splitting condition is met}
            \STATE Split cell $\mathcal{C}_i^t$ into sub-cells
        \ENDIF
    \ENDFOR
\ENDFOR
\end{algorithmic}
\end{algorithm}

\subsection{Computational Complexity Analysis}

In each time slot \( t \), the algorithm executes several key steps, whose computational cost is analyzed below. Let \( N \) denote the number of available workers, \( M \) the number of tasks, \( d \) the dimension of the context vector, and \( N_{\text{cell}}^t \) the number of partitions (cells) in the context space at time \( t \). The total number of possible arms (i.e., worker-task pairs) is given by \( K = N \cdot M \).

\begin{itemize}
    \item \textbf{Context Construction:} For each arm $(w_i, T_j)$, a $d$-dimensional context vector $\mathbf{z}_{i,j}^t$ is constructed. This results in a complexity of $\mathcal{O}(N \cdot M \cdot d)$.

    \item \textbf{Cell Identification and Profit Estimation:} For each context vector, the algorithm identifies the corresponding cell in the partitioned context space. Assuming a linear scan over $N_{cell}^t$ cells, the total cost is $\mathcal{O}(N \cdot M \cdot P_t)$. Within each cell, the estimated reward $\hat{Q}(\mathbf{z}_{i,j}^t)$, cost $C_{i,j}^t$, and profit $\hat{\Pi}_{i,j}^t$ are computed in constant time.

    \item \textbf{Task Assignment Optimization:} Based on the estimated profits, the platform solves a one-to-one task assignment problem by computing a maximum-weight bipartite matching on an $N \times M$ profit matrix. Since we assume $N > M$, we pad the matrix with $N - M$ virtual tasks (with zero profit) to form a square $N \times N$ matrix. We then apply the Hungarian algorithm, which ensures globally optimal assignment with a worst-case time complexity of $\mathcal{O}(N^3)$.

    \item \textbf{Feedback Update and Cell Refinement:} After task execution, only the partitions corresponding to selected arms are updated. For each affected cell, the empirical mean $\hat{\mu}$ and confidence term $c$ are updated. If the splitting condition is met, the cell is divided into sub-cells. In the worst case, all $N_{\text{cell}}^t$ cells may require updates, leading to a cost of $\mathcal{O}(P_t)$ per round.

\end{itemize}

\noindent
Combining all steps, the total per-round time complexity is:

\[
\mathcal{O}(N \cdot M \cdot d + N \cdot M \cdot N_{cell}^t + N^3 + N_{cell}^t)
\]

\subsection{Regret Analysis}

We provide a theoretical analysis of our adaptive contextual bandit algorithm with cell partitioning. Let $T$ be the time horizon, $d$ the context dimension, and $\rho=1/2$ the splitting parameter.

\subsubsection{Algorithm Properties}
\begin{itemize}
    \item The context space $\mathcal{Z}\subseteq[0,1]^d$ is partitioned into hyper-rectangular cells
    \item Each cell $\mathcal{C}$ at depth $h$ has diameter $\rho^h=2^{-h}$
    \item Cell $\mathcal{C}$ splits when $c(\mathcal{C}) \leq \rho^h$ where $c(\mathcal{C})=\sqrt{\frac{\ln T}{N(\mathcal{C})}}$
    \item The UCB for vector $\mathbf{z}$ in cell $\mathcal{C}$ is:
    \[
    \hat{Q}(\mathbf{z}_i^t) = \hat{\mu}(\mathcal{C}_i^t) + c(\mathcal{C}_i^t) + \rho^h
    \]
\end{itemize}

\subsubsection{Key Lemmas}

\begin{lemma}[Concentration Bound]
For any cell $\mathcal{C}$ and time $t$, with probability $\geq 1-T^{-2}$:
\[
|\hat{\mu}(\mathcal{C}) - \mu(\mathcal{C})| \leq c(\mathcal{C})
\]
where $\mu(\mathcal{C})$ is the true mean reward for $\mathcal{C}$.
\end{lemma}

% 对cell的估计不会过分离谱地低于真实值太多(和refinement level有关)
\begin{lemma}[Optimal Arm Containment]
If \(\overset{*}{\mathbf{z}_i^t} \in \mathcal{C}_*\), then with high probability:
\[
\hat{Q}(\overset{*}{\mathbf{z}_i^t}) \geq Q(\mathbf{z}_i^t) - O(\rho^{h_*})
\]

\end{lemma}

\subsubsection{Regret Decomposition}

Let $\Delta_i^t = Q(\overset{*}{\mathbf{z}}_i^t) - Q(\mathbf{z}_i^t)$ be the instantaneous regret for arm $i$ at time $t$. Through the UCB selection rule and context partitioning, we decompose it as:

\begin{align}
\Delta_i^t &= \underbrace{Q(\overset{*}{\mathbf{z}}_i^t) - \hat{Q}(\overset{*}{\mathbf{z}}_i^t)}_{\text{(a) Optimal arm estimation}} \nonumber \\
          &+ \underbrace{\hat{Q}(\overset{*}{\mathbf{z}}_i^t) - \hat{Q}(\mathbf{z}_i^t)}_{\text{(b) Suboptimal selection}} \nonumber \\
          &+ \underbrace{\hat{Q}(\mathbf{z}_i^t) - Q(\mathbf{z}_i^t)}_{\text{(c) Selected arm estimation}} \label{eq:decomp}
\end{align}

where $\overset{*}{\mathbf{z}}_i^t$ denotes the optimal context vector in the same cell $\mathcal{C}_i^t$ containing $\mathbf{z}_i^t$. We bound each term:

\begin{itemize}
\item \textbf{(a)} By Lipschitz continuity and cell refinement ($\mathbf{z}_i^t, \overset{*}{\mathbf{z}}_i^t \in \mathcal{C}_i^t$):
\[
Q(\overset{*}{\mathbf{z}}_i^t) - \hat{Q}(\overset{*}{\mathbf{z}}_i^t) \leq L\rho^{h} + c(\mathcal{C}_i^t) = O(\rho^{h})
\]

\item \textbf{(b)} Non-positive due to UCB selection:
\[
\hat{Q}(\overset{*}{\mathbf{z}}_i^t) - \hat{Q}(\mathbf{z}_i^t) \leq 0 \quad \text{(since } \mathbf{z}_i^t = \arg\max_{\mathbf{z} \in \mathcal{C}_i^t} \hat{Q}(\mathbf{z}))
\]

\item \textbf{(c)} Controlled by confidence interval:
\[
\hat{Q}(\mathbf{z}_i^t) - Q(\mathbf{z}_i^t) \leq c(\mathcal{C}_i^t) + \rho^h = O\left(\sqrt{\frac{\ln T}{N(\mathcal{C}_i^t)}} + \rho^h\right)
\]
\end{itemize}

\noindent Thus, the per-arm regret satisfies:
\begin{equation}
\Delta_i^t \leq O\left(\rho^h + \sqrt{\frac{\ln T}{N(\mathcal{C}_i^t)}}\right) \label{eq:per_arm_bound}
\end{equation}

\subsubsection{Total Regret Analysis}
Let $M_t \leq M_{\text{max}}$ be the actual number of selected arms at time $t$. The total regret $R(T)$ is:

\begin{align}
R(T) &= \sum_{t=1}^T \sum_{i=1}^{M_t} \Delta_i^t \nonumber \\
     &\leq \sum_{t=1}^T \sum_{i=1}^{M_t} O\left(\rho^h + \sqrt{\frac{\ln T}{N(\mathcal{C}_i^t)}}\right) \nonumber \\
     &= \underbrace{\sum_{t=1}^T M_t \cdot O(\rho^h)}_{\text{Approximation error}} + \underbrace{\sum_{t=1}^T \sum_{i=1}^{M_t} O\left(\sqrt{\frac{\ln T}{N(\mathcal{C}_i^t)}}\right)}_{\text{Estimation error}} \label{eq:total_regret}
\end{align}

\noindent To prove sublinearity ($R(T) = o(T)$), we analyze both terms:

1. \textbf{Approximation Error}:  
   With adaptive partitioning ($\rho^h = 2^{-h}$), the worst-case sum is $O(T^{1/(d+2)})$.

2. \textbf{Estimation Error}:  
   By the splitting condition $c(\mathcal{C}_i^t) \leq \rho^h$, we have:
   \[
   \sum_{t,i} \sqrt{\frac{\ln T}{N(\mathcal{C}_i^t)}} \leq \tilde{O}\left(\sqrt{T}\right)
   \]

\noindent Combining these yields:
\begin{equation}
R(T) \leq \tilde{O}\left(T^{\frac{d+1}{d+2}}\right) \label{eq:final_bound}
\end{equation}




% \noindent
% In line with the semi-bandit feedback assumption, the platform receives individual task quality observations in the form of $Q(\mathbf{z}_i^t)$ for each selected arm. These context-dependent observations are used to update the empirical mean $\hat{\mu}(\mathcal{C}_i^t)$ and confidence bound $c(\mathcal{C}_i^t)$ associated with the corresponding region $\mathcal{C}_i^t$ in the context space. This feedback-driven update mechanism allows the system to progressively refine its value estimates and focus future exploration on high-potential regions.



\section{Experiments}
\label{sec:experiments}

\subsection{Setup}
We implement a custom Python simulation to emulate a crowd-powered video analytics platform. All experiments run on a local desktop (Intel Core i5-13400F, 10 cores/16 threads, 2.50--4.0\,GHz) without GPU acceleration. The simulator generates synthetic task streams, models heterogeneous mobile workers, and executes the full scheduling loop including context normalization, quality estimation, and assignment.

We compare four policies:
\textbf{Ours} (our method, Hungarian matching with adaptive partitioning),
\textbf{Greedy} (ranking candidates by estimated reward and greedily selecting non-conflicting pairs),
\textbf{Random} (baseline without learning), and an \textbf{Oracle} that knows the ground-truth success probability $p(x)$ and serves as an upper bound.
All methods share the same pre-generated task streams and reward samples (Bernoulli draws with a fixed seed) to ensure a fair comparison.

Unless otherwise stated, we enable worker dynamics so that the worker pool evolves during training. This highlights the ability of Ours to adapt to changing supply while maintaining a stable advantage over Greedy/Random.

\subsection{Workload and Environment}
The simulated platform contains a task queue, a dynamic worker pool (initially 10 workers), and a context normalizer. Each round consists of new task arrivals, candidate generation, matching, execution, and online model updates.

\textbf{Task arrival:} At step $t$ we sample $N_t \sim \mathcal{U}\{6, \ldots, 15\}$ tasks. Each task carries attributes such as type, data size, deadline, and implicit quality preferences. We run $T=1000$ time steps in the main experiments.

\textbf{Worker features:} Workers are characterised by driving speed, bandwidth, processor speed, physical distance, task type compatibility, data size tolerance, and weather robustness. All features are normalized into $[0,1]$ using the ranges in \texttt{WORKER\_FEATURE\_VALUES\_RANGE}.

\textbf{Worker dynamics:} At every step each worker independently leaves with probability $0.03$. New workers join with probability $0.15$ and a batch size sampled from $\{1,2,3\}$. This keeps the expected number of active workers close to ten, mimicking a moderately dynamic deployment.

\subsection{Methods}

\textbf{Ours (proposed):} We maintain an adaptive partition tree over the context space. Each leaf stores a Beta posterior with prior $\alpha_0,\beta_0$ and observed successes/failures. When the posterior variance remains high after observing \texttt{PARTITION\_MIN\_SAMPLES} assignments, the leaf is split along the longest dimension. At decision time, we compute the posterior mean (optionally plus UCB bonus) minus replication cost to form a net reward matrix and solve a one-to-one matching with the Hungarian algorithm. Only positive-net pairs are retained.

\textbf{Greedy (baseline):} Uses the same posterior estimates as Ours but ranks candidate pairs in descending order of estimated net reward and greedily picks non-conflicting pairs until no positive-net option remains. This baseline reflects the widely-used ``max score per assignment'' heuristic.

\textbf{Random (baseline):} Filters out non-positive pairs, randomly shuffles the rest, and greedily selects non-conflicting pairs.

\textbf{Oracle (upper bound):} Uses the ground-truth $p(x)$ to fill the net reward matrix ($p(x)-c$) and runs the Hungarian algorithm. Oracle does not learn; it only serves as an upper bound in the plots.

\subsection{Success Probability and Reward}
Given a normalized context vector $x \in [0,1]^d$, the success probability $p(x)$ is simulated using a smooth, nonlinear function that includes:

- Diminishing marginal returns (via square root) and pairwise interactions (via geometric mean) for \textbf{positive features} such as driving speed, bandwidth, and processor performance;
- Convex penalties for \textbf{negative features} such as distance, data size, and weather;
- A soft gating effect modeled by a sigmoid function to simulate bottleneck behavior.

The execution reward is drawn from a Bernoulli distribution $\text{Bernoulli}(p(x))$ per assignment. Each assignment incurs a unit replication cost of $c=0.1$, and the net reward is defined as reward minus cost.

\subsection{Metrics}

\textbf{Loss (regret) per step:} Taking the oracle’s expected net reward as the upper bound, the step loss is defined as
\[
\text{loss}_t = \max\left(0, \ \mathbb{E}[\text{net}]_{\text{oracle}} - \mathbb{E}[\text{net}]_{\text{alg}}\right).
\]

\textbf{Cumulative Net Reward:} The total net reward is computed as
\[
\sum_{t} (\text{reward}_t - c),
\]
where $\text{reward}_t$ is the sampled binary success feedback drawn from $p(x)$.

To ensure fairness, all algorithms run independently on the same task stream, and their performance curves are plotted accordingly.

\subsection{Hyperparameters}

The default hyperparameters used in the experiments are:

\begin{itemize}
    \item Replication cost: $c = 0.2$ (\texttt{REPLICATION\_COST})
    \item Maximum replicas per task: $B = 1$ (\texttt{budget})
    \item Partition split threshold: $\tau = 10$ (\texttt{partition\_split\_threshold})
    \item Minimum samples before split: $6$ (\texttt{PARTITION\_MIN\_SAMPLES})
    \item Variance threshold for split: $0.01$ (\texttt{PARTITION\_VARIANCE\_THRESHOLD})
    \item Maximum partition depth: $D = 64$ (\texttt{MAX\_PARTITION\_DEPTH})
    \item Split strategy: longest dimension (\texttt{PARTITION\_SPLIT\_STRATEGY=longest})
    \item Batch size: 10 (\texttt{COMPARISON\_BATCH\_SIZE})
    \item Total steps: 1000 (\texttt{COMPARISON\_STEPS})
    \item Random seed: 43 (\texttt{RANDOM\_SEED})
\end{itemize}

\subsection{Results}

Figure~\ref{fig:loss} plots the per-step loss (regret). Ours quickly drives the regret below both Greedy and Random, confirming that adaptive partitioning accelerates convergence even under worker arrivals/departures. For a clearer long-term trend we also report a smoothed view (Figure~\ref{fig:loss_smooth}) obtained via a rolling mean (window $=50$) together with a 10--90\% quantile band; the smoothed curve shows Ours consistently maintaining the lowest regret plateau.

\begin{figure}[t]
\centering
\includegraphics[width=0.95\linewidth]{VideoAnalytics/fig/compare_loss.png}
\caption{Per-step loss. Ours vs. Greedy vs. Random.}
\label{fig:loss}
\end{figure}

\begin{figure}[t]
\centering
\includegraphics[width=0.95\linewidth]{VideoAnalytics/fig/compare_loss_smooth.png}
\caption{Smoothed per-step loss (rolling mean window $=50$, shaded area: 10--90\% percentile band).}
\label{fig:loss_smooth}
\end{figure}

Figure~\ref{fig:cum} further shows that Ours accumulates the highest net reward realized and tracks the Oracle upper bound much more closely than the baselines.

\begin{figure}[t]
\centering
\includegraphics[width=0.95\linewidth]{VideoAnalytics/fig/compare_cum_reward.png}
\caption{Realized cumulative net reward. Ours remains ahead of Greedy/Random while approaching Oracle.}
\label{fig:cum}
\end{figure}

\paragraph{Partition refinement.} To visualise the evolution of the adaptive context tree, Figure~\ref{fig:partitions} shows the partition boundaries in the two-dimensional (speed, bandwidth) space at several training steps. The partition is initially coarse; as more assignments are observed the tree refines around frequently visited regions and eventually concentrates on the high-speed/high-bandwidth quadrant where many profitable matches occur.

\begin{figure}[t]
\centering
\includegraphics[width=0.95\linewidth]{VideoAnalytics/fig/partition 可视化.png}
\caption{Evolution of the adaptive partition over two-dimensional contexts. Boundaries densify in frequently visited regions and converge to the high-speed/high-bandwidth quadrant.}
\label{fig:partitions}
\end{figure}

\paragraph{Challenge scenario.}\label{sec:challenge}
We further construct a conflict-heavy benchmark where different task clusters strongly favor different worker specialties. In this regime, Greedy often allocates the same “universal” workers to multiple clusters and leaves other tasks under-served, while Ours uses the Hungarian solver to coordinate assignments globally. Figure~\ref{fig:challenge_totals} shows that Ours accumulates a larger reward margin over Greedy in this setting.

\begin{figure}[t]
\centering
\includegraphics[width=0.95\linewidth]{VideoAnalytics/fig/challenge_totals.png}
\caption{Conflict-heavy benchmark (Section~\ref{sec:challenge}): Ours secures a larger margin over Greedy.}
\label{fig:challenge_totals}
\end{figure}

\subsection{Ablations and Reproducibility}

\textbf{Worker Dynamics.} All reported experiments enable the dynamic-arrival setting (leave probability 0.03, join probability 0.15 with batches of 1–3). For ablations we additionally run a static variant by disabling dynamics.

\textbf{Task Stream and Randomness.} The task stream is pre-generated and shared across all methods. Randomness in reward sampling is controlled using a fixed global seed.

\textbf{Reproducibility.} The experiments can be reproduced by running the main script \texttt{scheduler.py} with \texttt{RUN\_COMPARISON=True}. Parameters such as step count, batch size, and arrival range can be adjusted in \texttt{config.py}.


\section{Conclusion}

In this paper, we proposed an adaptive contextual task assignment framework for edge-assisted crowdsourcing systems in vehicular networks, with a focus on energy-aware and reward-optimized scheduling of video analytics tasks. By modeling worker-task pairs via context vectors and employing an online contextual bandit approach with dynamic partition refinement, our method is capable of learning high-reward allocations under uncertainty.

We developed a simulation environment to evaluate the proposed algorithm and conducted comparative experiments against baseline strategies. The results demonstrate that our method achieves significantly higher cumulative net reward and lower per-step regret compared to the random baseline, and approaches the performance of an oracle strategy with full knowledge of true rewards.

Although our experiments are based on synthetic data and simplified assumptions, the promising results suggest the effectiveness of adaptive context modeling in dynamic edge environments. In future work, we plan to incorporate more realistic datasets, heterogeneous mobility models, and dynamic worker availability, and extend our method to support multi-task replication under energy and latency constraints.



\bibliographystyle{IEEEtran}
\bibliography{ref}

\end{document}
